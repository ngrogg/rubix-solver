\documentclass{article}
\usepackage{amsmath, amssymb, amsthm, graphicx, enumerate, siunitx, textgreek}
\usepackage{hyperref}
\hypersetup{
    colorlinks=true,
    urlcolor=blue}


\newtheorem{theorem}{Theorem}%[section]
\newtheorem{lemma}[theorem]{Lemma}
\newtheorem{proposition}[theorem]{Proposition}
\newtheorem{corollary}[theorem]{Corollary}
\newtheorem{axiom}{Axiom}
\newtheorem*{remark}{Remark}
\newtheorem*{definition}{Definition}

\title{Rubik's Cube Group}
\author{Akil Marshall}
\begin{document}
\maketitle
\section{Introduction}
It is my goal to apply group theory and search techniques from A.I. to study techniques to solve and and solutions to the rubik's cube.

Currently the \href{https://en.wikipedia.org/wiki/Rubik\%27s\_Cube\_group}{wikipedia page} and this \href{https://www.quora.com/Group-Theory-mathematics/Why-are-commutators-useful-for-solving-permutation-puzzles/answer/Mark-Eichenlaub}{article} are my primary research leads.

Ultimately I would like to be able to build an environment that can easily facilitate the development of differnt solution seeking methods.
\end{document}
