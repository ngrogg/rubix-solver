\documentclass{article}
\usepackage{amsmath, amssymb, amsthm, graphicx, enumerate, siunitx, textgreek}
\usepackage{hyperref}
\hypersetup{
    colorlinks=true,
    urlcolor=blue}


\newtheorem{theorem}{Theorem}%[section]
\newtheorem{lemma}[theorem]{Lemma}
\newtheorem{proposition}[theorem]{Proposition}
\newtheorem{corollary}[theorem]{Corollary}
\newtheorem{axiom}{Axiom}
\newtheorem*{remark}{Remark}
\newtheorem*{definition}{Definition}

\title{Rubik's Cube Group}
\author{Akil Marshall}
\begin{document}
\maketitle
\section{Introduction}
The goal of this document is to build up a framework that links group theory to techniques in artificial intelligence, first search techniques, with luck more sophisticated approaches later.

\section{What is a Group}
A group is a set of objects with method of combining them called the group operation $\langle G,\cdot\rangle$. The set cannot be empty and the group operation must obey 4 conditions:
\begin{itemize}
    \item For all $a,b\in G$,
        \begin{align*}
            a\cdot b\in G
        \end{align*}
        The group operation is \textbf{closed} in $G$.
    \item For all $a,b,c\in G$,
        \begin{align*}
            a\cdot(b\cdot c)=(a\cdot b)\cdot c.
        \end{align*}
        The group operation is \textbf{associative} in $G$.
    \item There exists $e\in G$ such that for any $a\in G$,
        \begin{align*}
            a\cdot e=a=e\cdot a.
        \end{align*}
        There exists an \textbf{identity} element of the binary operation in $G$.
    \item For all $a\in G$ there exists  $a^{-1}\in G$ such that
        \begin{align*}
            a\cdot a^{-1}=e=a^{-1}\cdot a.
        \end{align*}
        There exists \textbf{inverse} elements of the binary operation in $G$.
\end{itemize}
\section{The Rubik's Cube Group}
It turns out that the Rubik's cube puzzle forms a group.
The set consists of the physical manipulations you can do to the puzzle and the binary operation is the composition of those manipulations.
\footnote{The turns are also associative, suppose $\tau$ and $\sigma$ are some turns of the cube you'll find that $(\sigma\tau\tau^{-1})\sigma^{-1}=\sigma(\tau\tau^{-1}\sigma^{-1})$.}
to to 
Put plainly if you turn a turn a Rubik's cube it results in a cube (closure), the ``identity'' is not manipulating the cube, and each turn and be undone (inverse). With these realizations we will build up a theory to describe manipulations (elements of the groups).


Before continuing I would like to note that I am a math enthusiast and not a math professor and will be rigorous as I can. However I will not be proving everything I do.

\subsection{Notation}

\section{Sources}
\begin{itemize}
    \item \url{https://en.wikipedia.org/wiki/Rubik\%27s\_Cube\_group}
    \item \url{https://www.quora.com/Group-Theory-mathematics/Why-are-commutators-useful-for-solving-permutation-puzzles/answer/Mark-Eichenlaub}
    \item \url{http://kociemba.org/cube.htm}
\end{itemize}
\end{document}
